\documentclass[12pt,a4paper,french]{book}
\usepackage{xunicode}
\usepackage{polyglossia}
	 \setmainlanguage[variant=swiss,frenchpart=false]{french}

\usepackage{setspace}
	%\singlespacing
	\onehalfspacing
	%\doublespacing
    
\usepackage{hyperref}

\usepackage{csquotes}



\begin{document}
\author{Jane Austen\\ édité par Élodie Paupe}
\title{Pride and Prejudice}
\date{1817}


\chapter*{Introduction}

Orgueil et Préjugés (Pride and Prejudice) est un roman de la femme de lettres anglaise Jane Austen paru en 1813. Il est considéré comme l'une de ses œuvres les plus significatives et est aussi la plus connue du grand public.

Rédigé entre 1796 et 1797, le texte, alors dans sa première version (First Impressions), figurait au nombre des grands favoris des lectures en famille que l'on faisait le soir à la veillée dans la famille Austen. Révisé en 1811, il est finalement édité deux ans plus tard, en janvier 1813. Son succès en librairie est immédiat, mais bien que la première édition en soit rapidement épuisée, Jane Austen n'en tire aucune notoriété: le roman est en effet publié sans mention de son nom (\enquote{par l'auteur de Sense and Sensibility}) car sa condition de \enquote{femme de la bonne société} lui interdit de revendiquer le statut d'écrivain à part entière\footnote{Le contenu de l'introduction est tiré de la notice \href{https://fr.wikipedia.org/wiki/Orgueil_et_Préjugés}{Wikipedia}}.

Drôle et romanesque, le chef-d'œuvre de Jane Austen continue à jouir d'une popularité considérable, par ses personnages bien campés, son intrigue soigneusement construite et prenante, ses rebondissements nombreux, et son humour plein d'imprévu. Derrière les aventures sentimentales des cinq filles Bennet, Jane Austen dépeint fidèlement les rigidités de la société anglaise au tournant des xviiie et xixe siècles. À travers le comportement et les réflexions d'Elizabeth Bennet, son personnage principal, elle soulève les problèmes auxquels sont confrontées les femmes de la petite gentry campagnarde pour s'assurer sécurité économique et statut social. À cette époque et dans ce milieu, la solution passe en effet presque obligatoirement par le mariage : cela explique que les deux thèmes majeurs d'Orgueil et Préjugés soient l'argent et le mariage, lesquels servent de base au développement des thèmes secondaires.

Grand classique de la littérature anglaise, Orgueil et Préjugés est à l'origine du plus grand nombre d'adaptations fondées sur une œuvre austenienne, tant au cinéma qu'à la télévision. Depuis Orgueil et Préjugés de Robert Z. Leonard en 1940, il a inspiré quantité d'œuvres ultérieures: des romans, des films, et même une bande dessinée parue chez Marvel.

Dans son essai de 1954, Ten Novels and Their Authors, Somerset Maugham le cite en seconde position parmi les dix romans qu'il considére comme les plus grands. En 2013, Le Nouvel Observateur, dans un hors-série consacré à la littérature des xixe et xxe siècle, le cite parmi les seize titres retenus pour le xixe siècle, le considérant comme \enquote{peut-être le premier chef-d'œuvre de la littérature au féminin}\footnote{\enquote{Les chefs-d'œuvre de la littérature commentés par les écrivains d'aujourd'hui}, Le Nouvel Observateur, hors-série no 83, juin/juillet 2013, \enquote{La Bibliothèque idéale (2): xixe et xxe siècle}.}.

\end{document}