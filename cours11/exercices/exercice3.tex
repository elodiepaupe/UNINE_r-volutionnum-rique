\documentclass[11pt]{book}
\usepackage{xunicode}
\usepackage{polyglossia}
	\setmainlanguage[variant=swiss]{french}

\usepackage{thalie}

\usepackage{setspace}
	%\singlespacing
	\onehalfspacing
	% \doublespacing
	
\newcommand{\didright}[1]{{\flushright \emph{#1}\par\medskip\flushleft}}

\title{Le Barbier de Séville}
\author{Pierre-Augustin Caron de Beaumarchais}
\date{1775}                                           


\begin{document}
\maketitle 

\frontmatter
\chapter{Liste des personnages}
(Les habits des acteurs doivent être dans l’ancien costume espagnol.)

\begin{dramatis}
\character[desc={grand d’Espagne, amant inconnu de Rosine, paraît, au premier acte, en veste et culotte de satin ; il est enveloppé d’un grand manteau brun, ou cape espagnole ; chapeau noir rabattu, avec un ruban de couleur autour de la forme. Au deuxième acte : habit uniforme de cavalier, avec des moustaches et des bottines. Au troisième : habillé en bachelier, cheveux ronds, grande fraise au cou ; veste, culotte, bas et manteau d’abbé. Au quatrième acte, il est vêtu superbement à l’espagnole avec un riche manteau ; par-dessus tout, le large manteau brun dont il se tient enveloppé.}, cmd={comte}, drama={Le comte Almaviva}]{Comte ALMAVIVA}

\character[desc={médecin, tuteur de Rosine : habit noir, court, boutonné ; grande perruque ; fraise et manchettes relevées ; une ceinture noire ; et, quand il veut sortir de chez lui, un long manteau écarlate.}, cmd={bartholo}, drama={Bartholo}]{BARTHOLO}

\character[desc={jeune personne d’extraction noble, et pupille de Bartholo ; habillée à l’espagnole.}, cmd={rosine}, drama={Rosine}]{Rosine}

\character[desc={barbier de Séville : en habit de major espagnol. La tête couverte d’un rescille, ou filet ; chapeau blanc, ruban de couleur autour de la forme, un fichu de soie attaché fort lâche à son cou, gilet et haut-de-chausses de satin, avec des boutons et boutonnières frangés d’argent ; une grande ceinture de soie, les jarretières nouées avec des glands qui pendent sur chaque jambe ; veste de couleur tranchante, à grands revers de la couleur du gilet ; bas blancs et souliers gris.}, cmd={figaro}, drama={Figaro}]{FIGARO}

\character[desc={organiste, maître à chanter de Rosine : chapeau noir rabattu, soutanelle et long manteau, sans fraise ni manchettes.}, cmd={basile}, drama={Don Basile}]{DON BASILE}

\character[desc={vieux domestique de Bartholo}, cmd={jeunesse}, drama={La Jeunesse}]{LA JEUNESSE}

\character[desc={autre valet de Bartholo, garçon niais et endormi. Tous deux habillés en Galiciens ; tous les cheveux dans la queue ; gilet couleur de chamois ; large ceinture de peau avec une boucle ; culotte bleue et veste de même, dont les manches, ouvertes aux épaules pour le passage des bras, sont pendantes par derrière.}, cmd={eveille}, drame={L'Eveillé}]{L'Eveillé}

\character[desc={}, cmd={notaire}, drama={Un Notaire}]{NOTAIRE}

\character[desc={homme de justice, avec une longue baguette blanche à la main.}, cmd={alcade}, drama={Un Alcade}]{UN ALCADE}

\end{dramatis}

Plusieurs alguazils et valets, avec des flambeaux.

\begin{dida}
La scène est à Séville, dans la rue et sous les fenêtres de Rosine, au premier acte ; et le reste de la pièce dans la maison du docteur Bartholo.
\end{dida}

\newpage

\mainmatter

\act{}

\scene{}
\onstage{\comtename}

\comte[seul, en grand manteau brun et chapeau rabattu. Il tire sa montre en se promenant.]
    Le jour est moins avancé que je ne croyais. L’heure à laquelle elle a coutume de se montrer derrière sa jalousie est encore éloignée. N’importe ; il vaut mieux arriver trop tôt que de manquer l’instant de la voir. Si quelque aimable de la cour pouvait me deviner à cent lieues de Madrid,  arrêté tous les matins sous les fenêtres d’une femme à qui je n’ai jamais parlé, il me prendrait pour un Espagnol du temps d’Isabelle. — Pourquoi non ? Chacun court après le bonheur.  Il est pour moi dans le cœur de \rosinename. — Mais quoi ! suivre une femme à Séville, quand Madrid et la cour offrent de toutes parts des plaisirs si faciles ? — Et c’est cela même que je fuis ! Je suis las des conquêtes que l’intérêt, la convenance ou la vanité nous présentent sans cesse. Il est si doux d’être aimé pour soi-même ! et si je pouvais m’assurer sous ce déguisement… Au diable l’importun !
    
\scene{}
\onstage{\figaroname et \comtename, caché.}


\figaro[une guitare sur le dos attachée en bandoulière avec un large ruban ; il chantonne gaiement, un papier et un crayon à la main]
\begin{verse}
Bannissons le chagrin,\\
Il nous consume :\\
Sans le feu du bon vin\\
  Qui nous rallume,\\
Réduit à languir,\\
L’homme sans plaisir \\
Vivrait comme un sot,\\
Et mourrait bientôt…
\end{verse}

Jusque-là ceci ne va pas mal, hein, hein.

\begin{verse}
  Et mourrait bientôt.\\
Le vin et la paresse\\
Se disputent mon cœur…
\end{verse}


Eh non ! ils ne se le disputent pas, ils y règnent paisiblement ensemble…

\begin{verse}
Se partagent… mon cœur…
\end{verse}

Dit-on se partagent ?… Eh ! mon Dieu ! nos faiseurs d’opéras-comiques n’y regardent pas de si près. Aujourd’hui, ce qui ne vaut pas la peine d’être dit, on le chante.

\begin{dida)
(Il chante.)
\end{dida}

\begin{verse}
Le vin et la paresse\\ 
Se partagent mon cœur…
\end{verse}


Je voudrais finir par quelque chose de beau, de brillant, de scintillant, qui eût l’air d’une pensée.

\begin{dida}
(Il met un genou en terre et écrit en chantant.)
\end{dida}

Se partagent mon cœur :
Si l’une a ma tendresse…
L’autre fait mon bonheur.


Fi donc ! c’est plat. Ce n’est pas ça… Il me faut une opposition, une antithèse :
\begin{verse}
Si l’une… est ma maîtresse,\\
L’autre…
\end{verse}

Eh ! parbleu ! j’y suis…
\begin{verse}
L’autre est mon serviteur.
\end{verse}

Fort bien, Figaro !…
\begin{dida}
(Il écrit en chantant.)
\end{dida}
\begin{verse}
Le vin et la paresse\\
Se partagent mon cœur :\\
Si l’une est ma maîtresse,\\
L’autre est mon serviteur,\\
L’autre est mon serviteur,\\
L’autre est mon serviteur.
\end{verse}

Hein, hein, quand il y aura des accompagnements là-dessous, nous verrons encore, messieurs de la cabale, si je ne sais ce que je dis… \did{Il aperçoit le comte.} J’ai vu cet abbé-là quelque part.

\begin{dida}
(Il se relève.)
\end{dida}

\comte[à part.]
Cet homme ne m’est pas inconnu.

\figaro
Eh ! non, ce n’est pas un abbé ! Cet air altier et noble…

\comte
Cette tournure grotesque…

\figaro
Je ne me trompe point ; c’est le \comtename.

\comte
Je crois que c’est ce coquin de \figaroname.

\figaro
C’est lui-même, monseigneur.

\comte
Maraud ! si tu dis un mot…

\figaro
Oui, je vous reconnais ; voilà les bontés familières dont vous m’avez toujours honoré.

\comte
Je ne te reconnaissais pas, moi. Te voilà si gros et si gras…

\figaro
Que voulez-vous, monseigneur, c’est la misère.

\comte
Pauvre petit ! Mais que fais-tu à Séville ? Je t’avais autrefois recommandé dans les bureaux pour un emploi.

\figaro
Je l’ai obtenu, monseigneur, et ma reconnaissance…

\comte
Appelle-moi Lindor. Ne vois-tu pas, à mon déguisement, que je veux être inconnu ?

\figaro
Je me retire.


\comte
Au contraire. J’attends ici quelque chose, et deux hommes qui jasent sont moins suspects qu’un seul qui se promène. Ayons l’air de jaser. Eh bien, cet emploi ?


\figaro
Le ministre, ayant égard à la recommandation de Votre Excellence, me fit nommer sur-le-champ garçon apothicaire.


\comte
Dans les hôpitaux de l’armée ?


\figaro
Non, dans les haras d’Andalousie.


\comte[riant.]
Beau début !


\figaro
Le poste n’était pas mauvais, parce qu’ayant le district des pansements et des drogues, je vendais souvent aux hommes de bonnes médecines de cheval…


\comte
Qui tuaient les sujets du roi !


\figaro
Ah ! ah ! il n’y a point de remède universel ; mais qui n’ont pas laissé de guérir quelquefois des Galiciens, des Catalans, des Auvergnats.


\comte
Pourquoi donc l’as-tu quitté ?


\figaro
Quitté ? C’est bien lui-même ; on m’a desservi auprès des puissances.

L’Envie aux doigts crochus, au teint pâle et livide…

\comte
Oh ! grâce ! grâce, ami ! Est-ce que tu fais aussi des vers ? Je t’ai vu là griffonnant sur ton genou, et chantant dès le matin.


\figaro
Voilà précisément la cause de mon malheur, Excellence. Quand on a rapporté au ministre que je faisais, je puis dire assez joliment, des bouquets à Chloris, que j’envoyais des énigmes aux journaux, qu’il courait des madrigaux de ma façon ; en un mot, quand il a su que j’étais imprimé tout vif, il a pris la chose au tragique, et m’a fait ôter mon emploi, sous prétexte que l’amour des lettres est incompatible avec l’esprit des affaires.


\comte
Puissamment raisonné ! Et tu ne lui fis pas représenter…


\figaro
Je me crus trop heureux d’en être oublié, persuadé qu’un grand nous fait assez de bien quand il ne nous fait pas de mal.


\comte
Tu ne dis pas tout. Je me souviens qu’à mon service tu étais un assez mauvais sujet.


\figaro
Eh ! mon Dieu ! monseigneur, c’est qu’on veut que le pauvre soit sans défaut.


\comte
Paresseux, dérangé…


\figaro
Aux vertus qu’on exige dans un domestique, Votre Excellence connaît-elle beaucoup de maîtres qui fussent dignes d’être valets ?


\comte[riant.]
Pas mal. Et tu t’es retiré en cette ville ?


\figaro
Non, pas tout de suite.


\comte[l’arrêtant.]
Un moment… J’ai cru que c’était elle… Dis toujours, je t’entends de reste.


\figaro
De retour à Madrid, je voulus essayer de nouveau mes talents littéraires ; et le théâtre me parut un champ d’honneur…


\comte
Ah ! miséricorde !


\figaro
\did{Pendant sa réplique, le comte regarde avec attention du côté de la jalousie.}
En vérité, je ne sais comment je n’eus pas le plus grand succès, car j’avais rempli le parterre des plus excellents travailleurs ; des mains… comme des battoirs ; j’avais interdit les gants, les cannes, tout ce qui ne produit que des applaudissements sourds ; et d’honneur, avant la pièce, le café m’avait paru dans les meilleures dispositions pour moi. Mais les efforts de la cabale…


\comte
Ah ! la cabale ! monsieur l’auteur tombé.


\figaro
Tout comme un autre : pourquoi pas ? Ils m’ont sifflé ; mais si jamais je puis les rassembler…


\comte
L’ennui te vengera bien d’eux ?


\figaro
Ah ! comme je leur en garde, morbleu !


\comte
Tu jures ! Sais-tu qu’on n’a que vingt-quatre heures au palais pour maudire ses juges ?


\figaro
On a vingt-quatre ans au théâtre : la vie est trop courte pour user un pareil ressentiment.


\comte
Ta joyeuse colère me réjouit. Mais tu ne me dis pas ce qui t’a fait quitter Madrid.


\figaro
C’est mon bon ange, Excellence, puisque je suis assez heureux pour retrouver mon ancien maître. Voyant à Madrid que la république des lettres était celle des loups, toujours armés les uns contre les autres, et que, livrés au mépris où ce risible acharnement les conduit, tous les insectes, les moustiques, les cousins, les critiques, les maringouins, les envieux, les feuillistes, les libraires, les censeurs, et tout ce qui s’attache à la peau des malheureux gens de lettres, achevait de déchiqueter et sucer le peu de substance qui leur restait ; fatigué d’écrire, ennuyé de moi, dégoûté des autres, abîmé de dettes et léger d’argent ; à la fin convaincu que l’utile revenu du rasoir est préférable aux vains honneurs de la plume, j’ai quitté Madrid ; et, mon bagage en sautoir, parcourant philosophiquement les deux Castilles, la Manche, l’Estramadure, la Siera-Morena, l’Andalousie ; accueilli dans une ville, emprisonné dans l’autre, et partout supérieur aux événements ; loué par ceux-ci, blâmé par ceux-là, aidant au bon temps, supportant le mauvais, me moquant des sots, bravant les méchants, riant de ma misère et faisant la barbe à tout le monde ; vous me voyez enfin établi dans Séville, et prêt à servir de nouveau Votre Excellence en tout ce qu’il lui plaira de m’ordonner.


\comte
Qui t’a donné une philosophie aussi gaie ?


\figaro
L’habitude du malheur. Je me presse de rire de tout, de peur d’être obligé d’en pleurer. Que regardez-vous donc toujours de ce côté ?


\comte
Sauvons-nous.


\figaro
Pourquoi ?


\comte
Viens donc, malheureux ! tu me perds.

\didright{(Ils se cachent.)}

\scene{}
    
\backmatter

\tableofcontents
    
\end{document}