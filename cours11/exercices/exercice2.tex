\documentclass[11pt]{book}
\usepackage{xunicode}
\usepackage{polyglossia}
	\setmainlanguage[variant=swiss]{french}

\usepackage{thalie}


\title{Le Barbier de Séville}
\author{Pierre-Augustin Caron de Beaumarchais}
\date{1775}                                           


\begin{document}
\maketitle 

\frontmatter
\chapter{Liste des personnages}
(Les habits des acteurs doivent être dans l’ancien costume espagnol.)

\begin{dramatis}
\character[desc={grand d’Espagne, amant inconnu de Rosine, paraît, au premier acte, en veste et culotte de satin ; il est enveloppé d’un grand manteau brun, ou cape espagnole ; chapeau noir rabattu, avec un ruban de couleur autour de la forme. Au deuxième acte : habit uniforme de cavalier, avec des moustaches et des bottines. Au troisième : habillé en bachelier, cheveux ronds, grande fraise au cou ; veste, culotte, bas et manteau d’abbé. Au quatrième acte, il est vêtu superbement à l’espagnole avec un riche manteau ; par-dessus tout, le large manteau brun dont il se tient enveloppé.}, cmd={almaviva}, drama={Le comte Almaviva}]{comte ALMAVIVA}

\character[desc={médecin, tuteur de Rosine : habit noir, court, boutonné ; grande perruque ; fraise et manchettes relevées ; une ceinture noire ; et, quand il veut sortir de chez lui, un long manteau écarlate.}, cmd={bartholo}, drama={Bartholo}]{BARTHOLO}

\character[desc={jeune personne d’extraction noble, et pupille de Bartholo ; habillée à l’espagnole.}, cmd={Rosine}, drama={Rosie}]{ROSINE}

\character[desc={barbier de Séville : en habit de major espagnol. La tête couverte d’un rescille, ou filet ; chapeau blanc, ruban de couleur autour de la forme, un fichu de soie attaché fort lâche à son cou, gilet et haut-de-chausses de satin, avec des boutons et boutonnières frangés d’argent ; une grande ceinture de soie, les jarretières nouées avec des glands qui pendent sur chaque jambe ; veste de couleur tranchante, à grands revers de la couleur du gilet ; bas blancs et souliers gris.}, cmd={figaro}, drama={Figaro}]{FIGARO}

\character[desc={organiste, maître à chanter de Rosine : chapeau noir rabattu, soutanelle et long manteau, sans fraise ni manchettes.}, cmd={basile}, drama={Don Basile}]{DON BASILE}

\character[desc={vieux domestique de Bartholo}, cmd={jeunesse}, drama={La Jeunesse}]{LA JEUNESSE}

\character[desc={autre valet de Bartholo, garçon niais et endormi. Tous deux habillés en Galiciens ; tous les cheveux dans la queue ; gilet couleur de chamois ; large ceinture de peau avec une boucle ; culotte bleue et veste de même, dont les manches, ouvertes aux épaules pour le passage des bras, sont pendantes par derrière.}, cmd={eveille}, drame={L'Éveillé}]{L'Éveillé}

\character[desc={}, cmd={notaire}, drama={Un Notaire}]{NOTAIRE}

\character[desc={homme de justice, avec une longue baguette blanche à la main.}, cmd={alcade}, drama={Un Alcade}]{UN ALCADE}

\end{dramatis}

Plusieurs alguazils et valets, avec des flambeaux.

\begin{dida}
La scène est à Séville, dans la rue et sous les fenêtres de Rosine, au premier acte ; et le reste de la pièce dans la maison du docteur Bartholo.
\end{dida}
\end{document}