\documentclass[french,10pt]{beamer} 
	%handout permet d'afficher la présentation sans les effets
	%\setbeameroption{hide notes} % Only slides
	\setbeameroption{show notes} % les deux
	%\setbeameroption{show only notes} % Only notes
	
\usepackage{xunicode}
\usepackage{polyglossia}
	\setmainlanguage[variant=swiss]{french}
	
\usepackage{graphicx}

\usepackage{bookmark}
\usepackage{beamerthemesplit}
    \usetheme{Singapore}
    \usecolortheme{seahorse}

\title[Cours 12]{TG: L'histoire du livre et sa troisième révolution (numérique)\\
    cours 12: dernier atelier}
\author[EP]{Élodie Paupe}
\institute{Master en littérature\\ semestre de printemps -- Université de Neuchâtel}
\date{\today}
\logo{\includegraphics[height=1cm]{UNINE_gris_pos.jpg}}
%\begin{frame}[plain] permet d’ôter les en-têtes et pieds de page.
%\begin{frame}[allowframebreaks] va couper un texte trop long et mettre la suite sur la page suivante.
%\setbeamertemplate{navigation symbols}{} enlève les liens de navigation
    
    \AtBeginSection[] %pour faire apparaître le plan au début de chaque section
    {
      \begin{frame}
      \frametitle{Plan de la séance}
      \tableofcontents[currentsection, hideothersubsections]
      \end{frame} 
    }


\begin{document}

\begin{frame}
    \maketitle
\end{frame}

\section{Introduction}

\begin{frame}
    Ici, la première diapositive après celle de titre
\end{frame}

\begin{frame}
    Ici un premier paragraphe
    \pause
    
    Puis un deuxième
    \pause 
    
    Puis un troisième
	\pause
	
	Puis du texte et de l'image.
	\includegraphics[height=1cm]{UNINE_gris_pos.jpg}    
    
\end{frame}

\section{Première partie}

\begin{frame}
    \onslide<2>
    Ici un premier paragraphe
    
    \onslide<1->
    Puis un deuxième
    
    \onslide<3->
    Puis un troisième
	
	\onslide<4>
	Puis du texte et de l'image.
	
	\includegraphics[height=1cm]{UNINE_gris_pos.jpg}    
    
\end{frame}

\begin{frame}
    \begin{itemize}
        \item<1-> Premier élément
        \item<2> Deuxième élément
        \item<3> Troisième élément
    \end{itemize}
\end{frame}

\begin{frame}
    \begin{itemize}[<+->]
        \item Premier élément
        \item \alert<3>{Deuxième élément}
        \item Troisième élément
    \end{itemize}
\end{frame}
\note{
Ici un petit blabla de note.}

\end{document}