%!TEX program=xelatex
\documentclass[12pt,a4paper]{book}
\usepackage{xunicode}
\usepackage{polyglossia}
	 \setmainlanguage[variant=swiss,frenchpart=false]{french}
    		\setotherlanguage{latin}
    		\renewenvironment{latin}
    	{\begin{hyphenrules}{latin}}
    	{\end{hyphenrules}}

\setlength{\parindent}{0pt}

\usepackage{reledmac}
\usepackage{reledpar}


\usepackage{setspace}
	%\singlespacing
	\onehalfspacing
	% \doublespacing

\begin{document}
\author{Thomas More}
\title{Utopie}
\date{}

\maketitle


\begin{pages}
    \begin{Leftside} 
    \begin{latin}
    \beginnumbering
    \pstart
    Sermo Raphaelis Hythlodaei uiri eximii de optimo reipublicae statu per illustrem uirum Thomam Morum inclitae Britanniarum urbis Londini et ciuem et uicecomitem. 
Cum non exigui momenti negotia quaedam inuictissimus Angliae Rex Henricus eius 
nominis octauus, omnibus egregii principis artibus ornatissimus, cum serenissimo 
castellae principe Carolo controuersa nuper habuisset, ad ea tractanda, 
componendaque, oratorem me legauit in Flandriam, comitem et collegam uiri 
incomparabilis Cuthberti Tunstalli, quem sacris scriniis nuper ingenti omnium 
gratulatione praefecit, de cuius sane laudibus nihil a me dicetur, non quod 
uerear ne parum sincerae fidei testis habenda sit amicitia, sed quod uirtus 
eius, ac doctrina maior est, quam ut a me praedicari possit, tum notior ubique 
atque illustrior, quam ut debeat, nisi uideri uelim solem lucerna, quod aiunt, 
ostendere. 
\pend
\pstart
Occurrerunt nobis Brugis - sic enim conuenerat - hi, quibus a principe negotium 
demandabatur, egregii uiri omnes. in his praefectus Brugensis uir magnificus, 
princeps et caput erat, ceterum os et pectus Georgius Temsicius Cassiletanus 
Praepositus, non arte solum, uerum etiam natura facundus, ad haec 
iureconsultissimus, tractandi uero negotii cum ingenio, tum assiduo rerum usu 
eximius artifex. ubi semel atque iterum congressi, quibusdam de rebus non satis 
consentiremus, illi in aliquot dies uale nobis dicto, Bruxellas profecti sunt, 
principis oraculum sciscitaturi. 
\pend
\pstart
Ego me interim - sic enim res ferebat - Antuerpiam confero. ibi dum uersor, 
saepe me inter alios, sed quo non alius gratior, inuisit Petrus Aegidius 
Antuerpiae natus, magna fide, et loco apud suos honesto, dignus honestissimo, 
quippe iuuenis haud scio doctiorne, an moratior. est enim optimus et 
litteratissimus, ad haec animo in omnes candido, in amicos uero tam propenso 
pectore, amore, fide, adfectu tam sincero, ut uix unum aut alterum usquam 
inuenias, quem illi sentias omnibus amicitiae numeris esse conferendum. rara 
illi modestia, nemini longius abest fucus, nulli simplicitas inest prudentior, 
porro sermone tam lepidus, et tam innoxie facetus, ut patriae desiderium, ac 
laris domestici, uxoris, et liberorum, quorum studio reuisendorum nimis quam 
anxie tenebar - iam tum enim plus quattuor mensibus abfueram domo - magna ex 
parte mihi dulcissima consuetudine sua, et mellitissima confabulatione 
leuauerit. 
\pend
\pstart
Hunc cum die quadam in templo diuae Mariae, quod et opere pulcherrimum, et 
populo celeberrimum est, rei diuinae interfuissem, atque peracto sacro, pararem 
inde in hospitium redire, forte colloquentem uideo cum hospite quodam, uergentis 
ad senium aetatis, uultu adusto, promissa barba, penula neglectim ab humero 
dependente, qui mihi ex uultu atque habitu nauclerus esse uidebatur. 
At Petrus ubi me conspexit, adit ac salutat. respondere conantem seducit 
paululum, et uides inquit hunc! - simul designabat eum cum quo loquentem uideram 
- eum inquit iam hinc ad te recta parabam ducere. uenisset inquam pergratus mihi 
tua causa. immo, inquit ille, si nosses hominem, sua. nam nemo uiuit hodie 
mortalium omnium, qui tantam tibi hominum, terrarumque incognitarum narrare 
possit historiam. quarum rerum audiendarum scio auidissimum esse te. ergo inquam 
non pessime coniectaui. nam primo aspectu protinus sensi hominem esse nauclerum. 
atqui inquit aberrasti longissime; nauigauit quidem non ut Palinurus, sed ut 
Ulysses; immo uelut nempe Plato. 
\pend
\pstart
Raphael iste, sic enim uocatur gentilicio 
nomine Hythlodaeus, et latinae linguae non indoctus, et graecae doctissimus - 
cuius ideo studiosior quam Romanae fuit, quoniam totum se addixerat 
philosophiae; qua in re nihil quod alicuius momenti sit, praeter Senecae 
quaedam, ac Ciceronis extare latine cognouit - relicto fratribus patrimonio, 
quod ei domi fuerat - est enim Lusitanus - orbis terrarum contemplandi studio 
Amerigo Vespucio se adiunxit, atque in tribus posterioribus illarum quattuor 
nauigationum quae passim iam leguntur, perpetuus eius comes fuit, nisi quod in 
ultima cum eo non rediit. curauit enim atque adeo extorsit ab Amerigo, ut ipse 
in his xxiiii esset qui ad fines postremae nauigationis in castello 
relinquebantur. itaque relictus est, uti obtemperaretur animo eius, 
peregrinationis magis quam sepulchri curioso. quippe cui haec assidue sunt in 
ore, caelo tegitur qui non habet urnam, et undique ad superos tantumdem esse 
uiae. quae mens eius, nisi deus ei propitius adfuisset, nimio fuerat illi 
constatura. 
\pend
\pstart
Ceterum postquam digresso Vespucio multas regiones cum quinque castellanorum 
comitibus emensus est, mirabili tandem fortuna Taprobanen delatus, inde peruenit 
in Caliquit, ubi repertis commode Lusitanorum nauibus, in patriam denique 
praeter spem reuehitur. 
\pend
\endnumbering
    \end{latin}
    \end{Leftside}

    \begin{Rightside} 
        \beginnumbering
        \pstart
        L'invincible roi d'Angleterre, Henri, huitième du nom, prince d'un génie rare et supérieur, eut, il n'y a pas longtemps, un démêlé de certaine importance avec le sérénissime Charles, prince de Castille. Je fus alors député orateur en Flandre, avec mission de traiter et arranger cette affaire. J'avais pour compagnon et collègue l'incomparable Cuthbert Tunstall, qui a été élevé depuis à la dignité de maître des Archives royales aux applaudissements de tous. Je ne dirai rien ici à sa louange. Ce n'est pas crainte qu'on accuse mon amitié de flatterie ; mais sa science et sa vertu sont au-dessus de mes éloges, et sa réputation est si brillante que vanter son mérite serait, comme dit le proverbe, faire voir le soleil une lanterne à la main.
		\pend
		\pstart
        Nous trouvâmes à Bruges, lieu fixé pour la conférence, les envoyés du prince Charles, tous personnages fort distingués. Le gouverneur de Bruges était le chef et la tête de cette députation, et George de Thamasia, prévôt de Mont-Cassel, en était la bouche et le cœur. Cet homme, qui doit son éloquence moins encore à l'art qu'à la nature, passait pour un des plus savants jurisconsultes en matière d'État ; et sa capacité personnelle, jointe à une longue pratique des affaires, en faisaient un très habile diplomate. Déjà le congrès avait tenu deux séances, et ne pouvait convenir sur plusieurs articles. Les envoyés d'Espagne prirent alors congé de nous pour aller à Bruxelles, consulter les volontés du prince. 
        \pend
		\pstart
        Moi, je profitai de ce loisir, et j'allai à Anvers. Pendant mon séjour dans cette ville, je reçus beaucoup de monde ; mais aucune liaison ne me fut plus agréable que celle de Pierre Gilles, Anversois d'une grande probité. Ce jeune homme, qui jouit d'une position honorable parmi ses concitoyens, en mérite une des plus élevées, par ses connaissances et sa moralité, car son érudition égale la bonté de son caractère. Son âme est ouverte à tous ; mais il a pour ses amis tant de bienveillance, d'amour, de fidélité et de dévouement, qu'on pourrait le nommer, à juste titre, le parfait modèle de l'amitié. Modeste et sans fard, simple et prudent, il sait parler avec esprit, et sa plaisanterie n'est jamais blessante. Enfin, l'intimité qui s'établit entre nous fut si pleine d'agrément et de charme, qu'elle adoucit en moi le regret de ma patrie, de ma maison, de ma femme, de mes enfants, et calma les inquiétudes d'une absence de plus de quatre mois. 
		\pend
		\pstart
        Un jour, j'étais allé à Notre Dame, église très vénérée du peuple, et l'un de nos plus beaux chefs-d'œuvre d'architecture ; et après avoir assisté à l'office divin, je me disposais à rentrer à l'hôtel, quand tout à coup je me trouve en face de Pierre Gilles, qui causait avec un étranger, déjà sur le déclin de l'âge. Le teint basané de l'inconnu, sa longue barbe, sa casaque tombant négligemment à demi, son air et son maintien annonçaient un patron de navire. A peine Pierre m'aperçoit-il qu'il s'approche, me tire un peu à l'écart alors que j'allais lui répondre et me dit en désignant son compagnon : - Vous voyez cet homme ; eh bien! j'allais le mener droit chez vous. - Mon ami, répondis-je, il eût été le bienvenu à cause de vous. - Et même à cause de lui, répliqua Pierre, si vous le connaissiez. Il n'y a pas sur terre un seul vivant qui puisse vous donner des détails aussi complets et aussi intéressants sur les hommes et sur les pays inconnus. Or, je sais que vous êtes excessivement curieux de ces sortes de nouvelles. - Je n'avais pas trop mal deviné, dis-je alors, car, au premier abord, j'ai pris cet homme pour un patron de navire. - Vous vous trompiez étrangement ; il a navigué, c'est vrai ; mais ce n'a pas été comme Palinure. Il a navigué comme Ulysse, voire comme Platon.
		\pend
		\pstart
        Écoutez son histoire : Raphaël Hythloday (le premier de ces noms est celui de sa famille) connaît assez bien le latin, et possède le grec en perfection. L'étude de la philosophie, à laquelle il s'est exclusivement voué, lui a fait cultiver la langue d'Athènes, de préférence à celle de Rome. Il n'ignorait pas qu'en cette matière les latins n'ont rien laissé d'important sauf quelques passages de Sénèque et de Cicéron. Le Portugal est son pays. Jeune encore, il abandonna son patrimoine à ses frères ; et, dévoré de la passion de courir le monde, il s'attacha à la personne et à la fortune d'Améric Vespuce. Il n'a pas quitté d'un instant ce grand navigateur, pendant les trois derniers des quatre voyages dont on lit partout aujourd'hui la relation. Mais il ne revint pas en Europe avec lui. Améric, cédant à ses vives instances, lui accorda de faire partie des vingt-quatre hommes qui restèrent lors du dernier voyage à Castel, le point le plus éloigné qu'atteignit l'expédition. Il fut donc laissé sur ce rivage, suivant son désir ; car notre homme ne craint pas la mort sur la terre étrangère ; il tient peu à l'honneur de pourrir dans un tombeau ; et souvent il répète cet apophtegme : Le cadavre sans sépulture a le ciel pour linceul ; partout il y a un chemin pour aller à Dieu. Ce caractère aventureux pouvait lui devenir fatal, si la Providence divine ne l'eût protégé.
        \pend
        \pstart
        Quoi qu'il en soit, après le départ de Vespuce, il parcourut avec cinq de ses compagnons du Castel une foule de contrées, débarqua à Taprobane comme par miracle, et de là parvint à Calicut, où il trouva des vaisseaux portugais qui le ramenèrent dans son pays, contre toute espérance.
        \pend
        \endnumbering
    \end{Rightside}
 \end{pages}
 \Pages

\end{document}